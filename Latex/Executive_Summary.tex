Mini Project 1 tasked us with not only analyzing pre-existing correlations including the Homogenous Equilibrium Model (HEM), the Lockhart-Martinelli (LM) correlation, and the Friedal correlation, but also with re-optimizing the Lockhart-martinelli correlation and creating our own correlation in order to best fit the data provided. We utilized engineering equation solver in order to create tables to import required properties of air and water in the correlations and LaTex in order to format our final report.
 
Part 1 of the mini project required us to observe how well the HEM, the LM correlation, and the Friedal correlation predicted the pressure gradient provided. Utilizing equations learned in class and in the class notes, we iterated through the provided tables and values using Python in order to return correlated pressure gradient values that we could compare to the provided experimental values. We then recorded the mean absolute error (MAE), the mean error (ME), the root mean square (RMS), and the \(R^2\) value between the correlated pressure gradient and the experimental pressure gradient for each correlation.

Part 2 of the mini project required us to re-optimize the primary empirical parameter in the LM correlation to best fit the data provided. We identified that parameter as the variable \(C\) in the two phase multiplier equation for \(\phi_l^2\) and we fit it to the data by oscillating the value for \(C\) in python until the MAE between the re-optimized correlated pressure gradient and the experimental pressure gradient reached a minimum. We then solved for the me, the rms, and the \(R^2\) value for our re-optimization, while providing the minimum MAE we observed.

Lastly, part 3 of the mini project required us to create our own correlation in order to best fit the provided data. We had to optimize two constants in the friction factor term for the frictional pressure loss. In order to do this, we oscillated values of \(C_1\) and \(C_2\) with specific predicted bounds in Python until we found the two values that minimized the MAE between our correlated pressure gradient and the experimental pressure gradient. We then solved for the me, the rms, and the \(R^2\) value for our correlation, while providing the minimum MAE we observed for those values of \(C_1\) and \(C_2\).
