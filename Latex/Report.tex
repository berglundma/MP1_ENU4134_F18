% This is the latex file that will be the final report for this project
\documentclass[10pt]{article}

\begin{document}

\section{Part 3 Discussion}

\hspace{15pt} In the development of our own model for the pressure gradient based on equations 3.1 and 3.2, we considered the use of three different Reynolds numbers: \(Re_{lo}\), \(Re_l\), and \(Re_g\). 

\[\frac{dp}{dz}_{fric}=f\frac{GU_{sg}}{D}\]
\[f=f(Re)=C_1Re^{C_2}\]

For case 1: \(Re=Re_{lo}\). Therefore, equation 3.3 was used to calculate \(f\). To solve for optimal \(C_1^{lo}\) and \(C_2^{lo}\) values, these equations were written into Python. The mean absolute error (MAE) between the correlated pressure gradients and experimental pressure gradients was iteratively calculated for different values of \(C_1^{lo}\) and \(C_2^{lo}\). The \(C_1^{lo}\) and \(C_2^{lo}\) values that yielded the smallest MAE were used as optimized values.
\[f=f_{lo} = C_1^{lo}Re_{lo}^{C_2^{lo}}\]


For case 2: \(Re=Re_l\). Therefore, equation 3.4 was used to calculate \(f\).
\[f=f_l = C_1^lRe_l^{C_2^l}\]


For case 3: \(Re=Re_g\). Therefore, equation 3.5 was used to calculate \(f\).
\[f=f_g = C_1^gRe_g^{C_2^g}\]


\end{document}
